\documentclass{article}

% if you need to pass options to natbib, use, e.g.:
%     \PassOptionsToPackage{numbers, compress}{natbib}
% before loading neurips_2021

% ready for submission
\usepackage[preprint]{neurips_2021}

% to compile a preprint version, e.g., for submission to arXiv, add add the
% [preprint] option:
%     \usepackage[preprint]{neurips_2021}

% to compile a camera-ready version, add the [final] option, e.g.:
%     \usepackage[final]{neurips_2021}

% to avoid loading the natbib package, add option nonatbib:
%    \usepackage[nonatbib]{neurips_2021}

\usepackage[utf8]{inputenc} % allow utf-8 input
\usepackage[T1]{fontenc}    % use 8-bit T1 fonts
\usepackage[colorlinks=true]{hyperref}       % hyperlinks
\usepackage{url}            % simple URL typesetting
\usepackage{booktabs}       % professional-quality tables
\usepackage{amsfonts}       % blackboard math symbols
\usepackage{nicefrac}       % compact symbols for 1/2, etc.
\usepackage{microtype}      % microtypography
\usepackage{xcolor}         % colors

\title{Spotify feat. Logistic Regression - \\ Popularity, Nothing Else Matters}

% The \author macro works with any number of authors. There are two commands
% used to separate the names and addresses of multiple authors: \And and \AND.
%
% Using \And between authors leaves it to LaTeX to determine where to break the
% lines. Using \AND forces a line break at that point. So, if LaTeX puts 3 of 4
% authors names on the first line, and the last on the second line, try using
% \AND instead of \And before the third author name.

\author{%
  Sebastian Hoffmann\\
  Matrikelnummer 5954377\\
  \texttt{sebastian.hoffmann@student.uni-tuebingen.de} \\
  \And
  Yannick Streicher\\
  Matrikelnummer 5331817\\
  \texttt{yannick.streicher@student.uni-tuebingen.de} \\
}

\begin{document}

\maketitle

\begin{abstract}
  What is the musical taste of the world? With the recent rise and global pervasiveness of music streaming services, such as Spotify, Deezer, or Apple Music, answering this question has become tractable. For this, we plan to analyze a \href{https://www.kaggle.com/rodolfofigueroa/spotify-12m-songs}{subset of 1.2 million songs} scraped from Spotify. However, this dataset lacks crucial information about popularity. Thus, an important step of our work is to augment the dataset further by querying the official Spotify REST API for a randomly sampled subset of the data. Besides a birds-eye overview of the musical landscape, e.g. distribution of genres, we want to identify common musical properties shared by popular songs, and likewise, very unpopular songs, using logistic regression. Such properties can be, for instance, tempo, mode, or key.
\end{abstract}


\end{document}
